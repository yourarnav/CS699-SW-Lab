\documentclass{article}
\usepackage{graphicx} % Required for including images
\usepackage{hyperref}
\usepackage{pdfpages}
\usepackage{placeins}
% Add space after each paragraph and indent the first line
\usepackage{parskip}
\setlength{\parskip}{1em}

\begin{document}

\title{\textit{CS 699} \\ \textit{From Theory to Application: Project Roadmap}}


\author{
 \textit{ Anuj Attri (23M0808)  Arnav Attri} \textit{(23M0811) }\\}


\date{\today} % Leave this blank to remove the date
\maketitle

\section*{{\fontsize{18}{30}\selectfont \textit{P}}roject Title }
% Title
\textit{  {Strokes Uncovered:} Data Analysis, Visualization, and Predictive Insights.}

\section{{\textit{\fontsize{18}{30}\selectfont S}olution Blueprint: An Insight into the Proposed Solutions}}

\begin{enumerate}
  \item {\textit{\textbf{Laying the Foundation}}: Configuring the Environment and Libraries.}
  \item {\textit{\textbf{Data Narratives}}: Unveiling Insights through EDA.}
  \item {\textit{\textbf{The Art of Data Cleaning}}: Best Practices in Preprocessing.}
  \item {\textit{\textbf{Balancing Act}}: Visualizing Data Imbalance and Sampling Techniques.}
  \item {\textit{\textbf{Beyond Accuracy}}: In-Depth Analysis of ML Model Evaluations.}
  \item {\textit{\textbf{Deploying the Stroke Model}}: A User-Friendly Web Application.}
 
\end{enumerate}

\section{{\textit{\fontsize{18}{30}\selectfont P}reliminary Results}}

Up to this point, our focus has been on configuring the environment and libraries and performing the \textit{Exploratory Data Analysis (EDA)}. We utilized \textit{\textbf{Plotly Express (px) and Seaborn}} to \textit{create} a range of \textit{\textbf{visualizations}} that provide us with valuable \textit{insights} into our dataset, enhancing our understanding of \textit{key} factors related to stroke prediction. Moreover, the \textit{results} of which are now available on \textit{our GitHub Repository} in the form of a \textit{Jupyter notebook}\footnote{\url{https://github.com/yourarnav/CS699-SW-Lab}}.\\


\section{{\textit{\fontsize{18}{30}\selectfont S}emester Roadmap: A Technical Journey Towards Excellence}}

After successfully navigating through the phase of \textit{Exploratory Data Analysis (EDA)}, we're now embarking on an exciting journey to address the \textit{significant challenges} in the field of \textit{stroke prediction and prevention}. In the coming weeks, we'll \textbf{\textit{focus}} on the following important technical milestones:

\begin{enumerate}
    \item \textbf{Data Preprocessing:} We'll meticulously refine and \textit{optimize our dataset}, ensuring that it's well-prepared for the subsequent modeling phase.
    
    \item \textbf{Balancing Data Visualization:} Using advanced \textit{data sampling techniques}, we aim to create a balanced dataset, guaranteeing fairness and accuracy in our analysis.
    
    \item \textbf{Modeling and Evaluation:} This is the \textit{high point of our expedition}, where we'll carefully build and fine-tune our \textit{machine learning models} to predict strokes with exceptional precision.
    
    \item \textbf{Model Deployment - Building a Web Application:} The \textit{grand finale} of our technical journey will be the creation of an \textit{interactive web application}. This application will allow users to directly experience the outcomes of our hard work and research.
\end{enumerate}

\end{document}
